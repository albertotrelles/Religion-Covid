\documentclass{article}

\title{Resultados}
\author{Alberto Trelles}
\date{\today}

\usepackage{graphicx} % For scalebox
\usepackage{ragged2e} % For justifying
\usepackage{array}    % For advanced table formatting
\usepackage{geometry} % Set margins
\usepackage{amssymb}  % For cross and checks 
\usepackage{pdflscape}
\usepackage{xcolor}   % For red text
\usepackage{booktabs} % For cmidrule
\usepackage{multirow}
\usepackage{multicol}


\begin{document}
\newgeometry{left=2.5cm, right=2.5cm, top=2.5cm, bottom=2.5cm}

\title{Religion as a coping mechanism: Evidence from Google Trends during lockdowns}

\author{Alberto Trelles\footnote{Trelles, Universidad del Pacífico.}} 


\maketitle

\begin{abstract}
Abstract here
\end{abstract}

\newpage

\section{Introduction}

Many coping mechanisms have been studied across various contexts as strategies to mitigate the effects of adverse shocks. In economics, attention has traditionally focused on material responses—such as saving, borrowing, or labor adjustments—but less is known about psychological coping mechanisms and their implications. The COVID-19 pandemic and implemented lockdowns represent a profound global shock, with widespread consequences for both physical and mental health. Emerging evidence suggests that lockdowns negatively impacted mental well-being. This raises questions about how individuals coped with the crisis. Among potential coping mechanisms, religion has drawn increasing attention. Correlational studies indicate that more religious individuals may have been better equipped to handle the psychological burden of the pandemic. This paper contributes to the literature by providing empirical evidence that individuals turned to religion as a coping mechanism during lockdowns, highlighting the role of non-material resources in responding to large-scale crises.



\section{Data}



\section{Identification strategy}







\end{document}